%%%%%%%%%%%%%%%%%%%%%%%%%%%%%%%%%%%%%%%%%%%%%%%%%
% Chapter: Introduction
%%%%%%%%%%%%%%%%%%%%%%%%%%%%%%%%%%%%%%%%%%%%%%%%%
\chapter{Introduction}

% ISSUE175: Make sure that in the end this introduction still includes the overall system description for HPC and the assumptions we are making (without forcing a particular architecture/implementation of PMIx)

\label{chap:intro}

\added{ \ac{PMIx} is an application programming interface specification to provide
High Performance Computing libraries and programming models with
portable and well defined access to commonly needed
distributed computing system software.
The \ac{PMIx} community strives to define interfaces to support the most widely
used HPC programming models and libraries.
\ac{PMIx} gives distributed systems software providers a better understanding of what
services can be provided to best support these programming models.
It also encourages the implementation of \ac{PMIx} APIs which allow for
portable access to these varied system software services and the
functionalities they offer.   Although these APIs can be implemented directly by the
system software components providing them, the \ac{PMIx} community
feels the development of centralized \ac{PMIx} implementations better serves the
HPC community.  Such implementations can interface
with existing system software or directly implement PMIx API's which are
not already provided by existing system software components.
\ac{PMIx} allows programming languages and libraries to focus on their core
competencies without having to provide their own services to provide
services that are better provided at a system level. }

\subsection{Background}
\label{chap:introduction:background}

\ac{PMI} has been used for quite some time as a means of exchanging wireup information needed for inter-process communication.  
Two versions (PMI-1 and PMI-2) have been released as part of the MPICH effort, with PMI-2 demonstrating better scaling properties than its PMI-1 predecessor.
\deleted{However, two significant challenges face the \ac{HPC} community as it continues to move towards machines capable of exaflop and higher performance levels: }

\begin{itemize}
\item \deleted{the physical scale of the machines, and the corresponding number of total processes they support, is expected to reach levels approaching  1 million processes executing across 100 thousand nodes. Prior methods for initiating applications relied on exchanging communication endpoint information between the processes, either directly or in some form of hierarchical collective operation. Regardless of the specific mechanism employed, the exchange across such large applications would consume considerable time, with estimates running in excess of 5-10 minutes; and}
\item \deleted{whether it be hybrid applications that combine OpenMP threading operations with MPI, or application-steered workflow computations, the HPC community is experiencing an unprecedented wave of new approaches for computing at exascale levels. One common thread across the proposed methods is an increasing need for orchestration between the application and the \ac{SMS} comprising the scheduler (a.k.a. the \ac{WLM}), the \ac{RM}, global file system, fabric, and other subsystems. The lack of available support for application-to-SMS integration has forced researchers to develop "virtual" environments that hide the SMS behind a customized abstraction layer, but this results in considerable duplication of effort and a lack of portability.}
\end{itemize}

\deleted{ \ac{PMIx} represents an attempt to resolve these questions by providing an extended version of the \ac{PMI} definitions specifically designed to support clusters up to exascale and larger sizes.} 
\replaced{ PMI-1 and PMI-2 can be implemented using \ac{PMIx} though it is not a strict superset of either.
Since its introduction, \ac{PMIx} has expanded 
on earlier \ac{PMI} efforts in the following ways:}{ The overall objective of the project is not to branch the existing definitions -- in fact, \ac{PMIx} fully supports both of the existing PMI-1 and PMI-2 \acp{API} -- but rather to:}

\begin{compactalphaenum}
\item \added{\ac{PMIx} provides an extended version of the \ac{PMI} API's which provide necessary functionality for launching and managing parallel applications at exascale.}
\item \added{The target user of \ac{PMIx} has grown from MPI libraries to other libraries and languages for HPC.}
\item \deleted{add flexibility to the existing \acp{API} by adding an array of key-value ``attribute'' pairs to each \ac{API} signature that allows implementers to customize the behavior of the \ac{API} as future needs emerge without having to alter or create new variants of it;}
\item \deleted{add new APIs that provide extended capabilities such as asynchronous event notification plus dynamic resource allocation and management;}
\item \deleted{establish a collaboration between \ac{SMS} subsystem providers including resource manager, fabric, file system, and programming library developers to define integration points between the various subsystems as well as agreed upon definitions for associated \acp{API}, attribute names, and data types;}
\item \deleted{form a standards-like body for the definitions; and}
\item \replaced{ A reference implementation (\ac{PRI}) of \ac{PMIx} has vastly increased adoption}
{provide a reference implementation of the \ac{PMIx} standard.}
\item \added{With the aid of the \ac{PRI}, several implementations of \ac{PMIx} API's have become available leading to further adoption.}
\end{compactalphaenum}

\added{The increase in adoption has motivated the creation of this document to formally specify the intended behavior of the \ac{PMIx} API's.}

\replaced{More}{Complete} information about the \ac{PMIx} standard and affiliated projects can be found at the \ac{PMIx} web site: \url{https://pmix.org}


%%%%%%%%%%%%%%%%%%%%%%%%%%%%%%%%%%%%%%%%%%%%%%%%%
%%%%%%%%%%%%%%%%%%%%%%%%%%%%%%%%%%%%%%%%%%%%%%%%%
\section{\deleted{Charter}}
\label{chap:intro:charter}

\deleted{The charter of the PMIx community is to:}
% ISSUE175:  Discussion around use of "PMIx community"... Specification?  Standard?
% ISSUE175: Some of these bullets could be put above as goals of the standard, some of these are goals of the overall effort?

\begin{itemize}
\item \deleted{Define a set of agnostic APIs (not affiliated with any specific programming model or code base) to support interactions between application processes and the \ac{SMS}.}
% ISSUE175: Remove (not needed anymore, was relevant at one time maybe)
\item \deleted{Develop an open source (non-copy-left licensed) standalone ``reference'' library implementation to facilitate adoption of the \ac{PMIx} standard.}
\item \deleted{Retain transparent backward compatibility with the existing PMI-1 and PMI-2 definitions, any future \ac{PMI} releases, and across all \ac{PMIx} versions.}
% ISSUE175: Not well defined.  Maybe a bit dated compared to all that PMIx does.
\item \deleted{Support the ``Instant On'' initiative for rapid startup of applications at exascale and beyond.}
% ISSUE175: If we redefine this section to a “purpose of the standard” then this bullet won’t work 
\item \deleted{Work with the \ac{HPC} community to define and implement new \acp{API} that support evolving programming model requirements for application interactions with the \ac{SMS}.}
\end{itemize}

% ISSUE175: Again community specific, so if we make this about this document, we need to remove this.
\deleted{ Participation in the \ac{PMIx} community is open to anyone, and not restricted to only code contributors to the reference implementation.}


%%%%%%%%%%%%%%%%%%%%%%%%%%%%%%%%%%%%%%%%%%%%%%%%%
%%%%%%%%%%%%%%%%%%%%%%%%%%%%%%%%%%%%%%%%%%%%%%%%%
\section{\deleted{PMIx Standard Overview}}
\label{chap:intro:std_overview}

% ISSUE175:  change "standard" to "specification" everywhere?

% ISSUE175:  This opening paragraph is pretty much exactly what we want to change in this PR
\deleted{ The \ac{PMIx} Standard defines and describes the interface developed by the \acf{PRI}.
Much of this document is specific to the \acf{PRI}'s design and implementation.
Specifically the standard describes the functionality provided by the \ac{PRI}, and what the \ac{PRI} requires of the clients and \acfp{RM} that use it's interface.}

%%%%%%%%%%%
\subsection{\deleted{Who should use the standard?}}

% ISSUE175: Not sure if we want to save anything from here?  

\deleted{The \ac{PMIx} Standard informs PMIx clients and \acp{RM} of the syntax and semantics of the \ac{PMIx} APIs.}

\deleted{\ac{PMIx} clients (e.g., tools, \ac{MPE} libraries) can use this standard to understand the set of attributes provided by various APIs of the \ac{PRI} and their intended behavior.
Additional information about the rationale for the selection of specific interfaces and attributes is also provided.}

\deleted{\ac{PMIx}-enabled \acp{RM} can use this standard to understand the expected behavior required of them when they support various interfaces/attributes.
In addition, optional features and suggestions on behavior are also included in the discussion to help guide \ac{RM} design and implementation.}

\section{\added{Terminology}}
\label{chap:intro:terminology}

\added{To be decided}
%%%%%%%%%%%
\subsection{\deleted{What is defined in the standard?}}

%ISSUE175:  Exactly what we want to change:

\deleted{The \ac{PMIx} Standard defines and describes the interface developed by the \acf{PRI}.
It defines the set of attributes that the \ac{PRI} supports; the set of attributes that are required of a \ac{RM} to support, for a given interface; and the set of optional attributes that an \ac{RM} may choose to support, for a given interface.}

%%%%%%%%%%%
\subsection{\deleted{What is \emph{not} defined in the standard?}}

% ISSUE175: Many of the terms in this section are not defined by this point in the document and do not make sense for 1st time readers.  

\added{ALL TEXT MOVED TO SECTION \ref{chap:intro:not_supported}}

%%%%%%%%%%%
\subsection{\deleted{General Guidance for PMIx Users and Implementors}}

% ISSUE175: Should this be the introduction to this chapter (current 1.1 obviously should not)?

% ISSUE175: PRI -> "of a PMIx implementation"
\deleted{The \ac{PMIx} Standard defines the behavior of the \acf{PRI}.}

\deleted{A complete system harnessing the \ac{PMIx} interface requires an agreement between the \ac{PMIx} client, be it a tool or library, and the \ac{PMIx}-enabled \ac{RM}.
The \ac{PRI} acts as an intermediary between these two entities by providing a standard API for the exchange of requests and responses.
% ISSUE175: PRI -> "of a PMIx implementation"
The degree to which the \ac{PMIx} client and the \ac{PMIx}-enabled \ac{RM} may interact needs to be defined by those developer communities.}
\deleted{ The \ac{PMIx} standard can be used to define the specifics of this interaction.}

% ISSUE175: The phrase PMIx-enabled RMs really makes no sense for this document

\added{ALL TEXT MOVED TO SECTION \ref{chap:intro:not_supported}}

\deleted{\ac{PMIx} clients (e.g., tools, \ac{MPE} libraries) may find that they depend only on a small subset of interfaces and attributes to work correctly.}
\deleted{\ac{PMIx} clients are strongly advised to define a document itemizing the \ac{PMIx} interfaces and associated attributes that are required for correct operation, and are optional but recommended for full functionality.}
\deleted{The \ac{PMIx} standard cannot define this list for all given \ac{PMIx} clients, but such a list is valuable to \acp{RM} desiring to support these clients.}

\deleted{\ac{PMIx}-enabled \acp{RM} may choose to implement a subset of the \ac{PMIx} standard and/or define attributes beyond those defined herein.
\ac{PMIx}-enabled \acp{RM} are strongly advised to define a document itemizing the \ac{PMIx} interfaces and associated attributes they support, with any annotations about behavior limitations.
The \ac{PMIx} standard cannot define this list for all given \ac{PMIx}-enabled \acp{RM}, but such a list is valuable to \ac{PMIx} clients desiring to support a broad range of \ac{PMIx}-enabled \acp{RM}.  }



%%%%%%%%%%%%%%%%%%%%%%%%%%%%%%%%%%%%%%%%%%%%%%%%%
%%%%%%%%%%%%%%%%%%%%%%%%%%%%%%%%%%%%%%%%%%%%%%%%%
\section{PMIx Architecture Overview}
\label{chap:intro:arch_overview}


% ISSUE175: Consider wholesale re-write with these goals:  
% 1) Define the terminology used throughout document
% 2) PMIx was designed to respect the archicture of  current HPC systems (e.g. separation between MPI and resource manager).  
% Do we want to use this opportunity to maintain that PMIx should focus on providing access to current functionality of HPC systems rather than re-architect existing functionality? 
% 3) PMIx to support distributed systems with a focus on the communication aspect.  

\replaced{The presentation of the \ac{PMIx} APIs within this document makes some 
basic assumptions about how these APIs
are used and implemented.  These assumptions are generally made only to simplify
the presentation and explain \ac{PMIx} with the expectation that most readers
have similar concepts on how computing systems are organized today.  However, ultimately
this document should only be assumed to define a set of APIs}{
This section presents a brief overview of the \ac{PMIx} Architecture~\cite{2017-Castain-EuroMPI}.
Note that this is a conceptual model solely used to help guide the standards process --- it does not represent
a design requirement on any \ac{PMIx} implementation. Instead, the model is used by the
\ac{PMIx} community as a sounding board for evaluating proposed interfaces and avoid unintentionally imposing
constraints on implementers. }

\replaced{
A concept that is fundamental to \ac{PMIx} is that a \ac{PMIx} implentation may 
operate primarily as a \textit{messenger}, and not a \textit{doer} --- i.e., a \ac{PMIx} 
implementation may rely heavily or fully on other software components to provide functionality.
Since a \ac{PMIx} implementation may only deliver requests and responses to other 
software components, the API calls include ways to provide arbitrary information to the 
backend components that actually 
implement the functionality.  Also, because the \ac{PMIx} implementations may rely heavily 
on other system software, a PMIx implementation may not be able to guarantee that a feature 
is available on all platforms the implementation supports.  These aspects are discussed in 
detail in the remainder of this chapter.
}{Built into the model are two guiding principles also reflected in the standard. First,
\ac{PMIx} operates in the mode of a \textit{messenger}, and not a \textit{doer} --- i.e., the role
of \ac{PMIx} is to provide communication between the various participants, relaying requests and returning
responses. The intent of the standard is not to suggest that \ac{PMIx} itself actually perform any of
the defined operations --- this is left to the various \ac{SMS} elements and/or the application. Any exceptions to that intent are left to the discretion of the particular implementation.}

\begingroup
\begin{figure*}[ht!]
  \begin{center}
    \includegraphics[clip,width=0.8\textwidth]{figs/PMIxRoles.pdf}
  \end{center}
  \caption{PMIx-SMS Interactions}
  \label{fig:roles}
\end{figure*}
\endgroup


\deleted{Thus, as the diagram in} Fig.~\ref{fig:roles} shows \replaced{a typical \ac{PMIx} implemention in which}{,} the application is built against a \ac{PMIx} client library that contains the client-side \acp{API},
attribute definitions, and communication support for interacting with the local \ac{PMIx} server. Intra-process cross-library interactions
are supported at the client level to avoid unnecessary burdens on the server. Orchestration requests are sent to the
local \ac{PMIx} server, which subsequently passes them to the host \ac{SMS} (here represented by an \ac{RM} daemon) using the \ac{PMIx} server callback functions the host \ac{SMS} registered during PMIx\_server\_init. The host \ac{SMS} can indicate its lack of support for any operation by simply providing a \textit{NULL} for the associated callback function, or can create a function entry that returns \textit{not supported} when called.

\added{ Maybe make the next paragraph an "advice to implementors" or "a quality PMIx implementation will attempt to consolidate requests from local PMIx clients into a single request to \ac{SMS} ecomponents when possible}

\deleted{The conceptual model places the burden of fulfilling the request on the host \ac{SMS}. This includes performing any
inter-node communications, or interacting with other \ac{SMS} elements. Thus, a client request for a network traffic report
does not go directly from the client to the \ac{FM}, but instead is relayed to the \ac{PMIx} server, and then passed to the host \ac{SMS}
for execution. This architecture reflects the second principle underlying the standard --- namely, that connectivity is to be minimized by channeling all application interactions with the \ac{SMS} through the local \ac{PMIx} server.}

%
% ISSUE175: Ralph C.:  What should we be taking away from this paragraph.  Wasn't clear to WG.
% Are there specific API's you can point us to to that would give us some context to understand?
%
Recognizing the burden this places on SMS vendors, the PMIx community has included interfaces by
which the host can request support from local SMS elements. Once the SMS has transferred the request to
an appropriate location, a PMIx server interface can be used to pass the request between SMS subsystems.
For example, a request for network traffic statistics can utilize the
PMIx networking abstractions to retrieve the information from the FM. This reduces the portability and
interoperability issues between the individual subsystems by transferring the burden of defining the
interoperable interfaces from the SMS subsystems to the PMIx community, which continues
to work with those providers to develop the necessary support.

% ISSUE175: Does the notes on how tools differ need to be brought up here?  It's not really critical t
% to the overall architecture, but it does break the notion of a client interacting with a local server

\replaced{\ac{PMIx} clients are processes which are started through the \ac{PMIx} infrastructure, 
either by the PMIx implementation directly or through an \ac{SMS} component, and have registered 
as clients using the \refapi{PMIx_Init} API.  A \ac{PMIx} client 
is created in such a way that when the process calls \refapi{PMIx_Init}, the \ac{PMIx} 
implementation will be have sufficient information available to
authenticate the caller.  The PMIx implementation will have sufficient knowledge about the  
process which it created, either directly or through other SMS, to be able to provide 
information the process requests such information about any peers it might have.   

Tools, whether standalone or embedded in job scripts, are an exception to 
the normal client registration process.   A process can register as a tool by 
providing adequate rendezvous information via \refapi{PMIx_tool_init}.  This allows processes
which were not created by the PMIx infrastructure to request access to PMIx functionality. 
Processes which register as tools do not have peers.  
}{Tools, whether standalone or embedded in job scripts, are an exception to 
the communication rule and can connect to
any PMIx server providing they are given adequate rendezvous information. The PMIx conceptual model views the
collection of PMIx servers as a cloud-like conglomerate --- i.e., orchestration and information requests can be
given to any server regardless of location. However, tools frequently execute on locations that may not house an
operating PMIx server --- e.g., a users notebook computer. Thus, tools need the ability to remotely connect to
the PMIx server ``cloud''.  }

\replaced{NEED A SUMMARY OF THIS SECTION}{
The scope of the PMIx standard therefore spans the range of these interactions, between client-and-SMS and between SMS
subsystems. Note again that this does not impose a requirement on any given PMIx implementation to cover the entire
range --- implementers are free to return \textit{not supported} from any PMIx function.
}

\section{\added{Some section about the use of ``Not Supported''}}
\label{chap:intro:not_supported}

\added{
It is difficult to define a portable API that will provice access to the many             
and varied features underlying the operations for which \ac{PMIx} provides access.
For example, the options and features provided to request the creation·
of new processes varied dramatically between different systems existing
at the time \ac{PMIx} was introduced.  Many \acp{WLM} provide rich interfaces·
to specify the resources assigned to new processes and how the new processes·
are mapped to those resources.  As a result, \ac{PMIx} is faced with the challenge·
of attempting to meet the seamingly conflicting goals of creating an API which allows·
access to these diverse features while being portable across a wide range of·
existing software environments. In addition, the functionalities required by different 
clients vary greatly.  Producing a \ac{PMIx} implementation
which can provide the needs of all possible clients on all of its target systems
could be so burdensome as to discourage \ac{PMIx} implementations.}

\deleted{No standards body can require an implementer to support something in their standard, and \ac{PMIx} is no different in that regard. While an implementer of the \ac{PMIx} library itself must at least include the standard \ac{PMIx} headers and instantiate each function, they are free to return ``not supported'' for any function they choose not to implement. }

% ISSUE175: Define Resource managers earlier, SMS

This also applies to the host environments. Resource managers and other system management stack components retain the right to decide on support of a particular function. The \ac{PMIx} community continues to look at ways to assist \ac{SMS} implementers in their decisions by highlighting functions that are critical to basic application execution (e.g., \refapi{PMIx_Get}), while leaving flexibility for tailoring a vendor's software for their target market segment.

One area where this can become more complicated is regarding the attributes that provide information to the client process and/or control the behavior of a \ac{PMIx} standard \ac{API}. For example, the \refattr{PMIX_TIMEOUT} attribute can be used to specify the time (in seconds) before the requested operation should time out. The intent of this attribute is to allow the client to avoid ``hanging'' in a request that takes longer than the client wishes to wait, or may never return (e.g., a \refapi{PMIx_Fence} that a blocked participant never enters).

% ISSUE175: Either change things like "library and its SMS host" to “PMIx implementation” or we need a real overview of the architecture so these things make more sense.

If an application (for example) truly relies on the \refattr{PMIX_TIMEOUT} attribute in a call to \refapi{PMIx_Fence}, it should set the required flag in the \refstruct{pmix_info_t} for that attribute. This informs the library and its \ac{SMS} host that it must return an immediate error if this attribute is not supported. By not setting the flag, the library and \ac{SMS} host are allowed to treat the attribute as optional, ignoring it if support is not available.

It is therefore critical that users and application implementers:

\begin{compactalphaenum}
\item consider whether or not a given attribute is required, marking it accordingly; and

\item check the return status on all \ac{PMIx} function calls to ensure support was present and that the request was accepted. Note that for non-blocking \acp{API}, a return of \refconst{PMIX_SUCCESS} only indicates that the request had no obvious errors and is being processed – the eventual callback will return the status of the requested operation itself.
\end{compactalphaenum}

\added{TEXT MOVED FROM OTHER SECTIONS:}

\ac{PMIx} clients (e.g., tools, \ac{MPE} libraries) may find that they depend only on a small subset of interfaces and attributes to work correctly.
\ac{PMIx} clients are strongly advised to define a document itemizing the \ac{PMIx} interfaces and associated attributes that are required for correct operation, and are optional but recommended for full functionality.
The \ac{PMIx} standard cannot define this list for all given \ac{PMIx} clients, but such a list is valuable to \acp{RM} desiring to support these clients.

\ac{PMIx}-enabled \acp{RM} may choose to implement a subset of the \ac{PMIx} standard and/or define attributes beyond those defined herein.
\ac{PMIx}-enabled \acp{RM} are strongly advised to define a document itemizing the \ac{PMIx} interfaces and associated attributes they support, with any annotations about behavior limitations.
The \ac{PMIx} standard cannot define this list for all given \ac{PMIx}-enabled \acp{RM}, but such a list is valuable to \ac{PMIx} clients desiring to support a broad range of \ac{PMIx}-enabled \acp{RM}. 

% ISSUE175: This paragraph would do better under a PMIx compliance section (where we define what that means).

While a \ac{PMIx} library implementer, or an \ac{SMS} component server, may choose to support a particular \ac{PMIx} \ac{API}, they are not required to support every attribute that might apply to it. This would pose a significant barrier to entry for an implementer as there can be a broad range of applicable attributes to a given \ac{API}, at least some of which may rarely be used. The \ac{PMIx} community is attempting to help differentiate the attributes by indicating those that are generally used (and therefore, of higher importance to support) vs those that a ``complete implementation'' would support.

Note that an environment that does not include support for a particular attribute/\ac{API} pair is not ``incomplete'' or of lower quality than one that does include that support. Vendors must decide where to invest their time based on the needs of their target markets, and it is perfectly reasonable for them to perform cost/benefit decisions when considering what functions and attributes to support.

The flip side of that statement is also true: Users who find that their current vendor does not support a function or attribute they require may raise that concern with their vendor and request that the implementation be expanded. Alternatively, users may wish to utilize the \ac{PRRTE} as a ``shim'' between their application and the host environment as it might provide the desired support until the vendor can respond. Finally, in the extreme, one can exploit the portability of PMIx-based applications to change vendors.

%%%%%%%%%%%
\subsection{\deleted{The \acf{PRI}}}

% ISSUE175: Small group on today’s call feel 1.3.1 and 1.3.2 need to be moved to referencåe implementation User’s Guide.   Whether to include a small statement recognizing the existence and value of the reference implementation is up for debate.

\deleted{The \ac{PMIx} community has committed to providing a complete, reference implementation of each version of the standard. Note that the definition of the \ac{PMIx} Standard is not contingent upon use of the \acf{PRI} --- any implementation that supports the defined \acp{API} is a \ac{PMIx} Standard compliant implementation. } 
\deleted{The \ac{PRI} is provided solely for the following purposes:}
\begin{itemize}
\item \deleted{Validation of the standard.\\
No proposed change and/or extension to the \ac{PMIx} standard is accepted without an accompanying prototype implementation in the \ac{PRI}.
This ensures that the proposal has undergone at least some minimal level of scrutiny and testing before being considered.}
\item \deleted{Ease of adoption.\\
The \ac{PRI} is designed to be particularly easy for resource managers (and the \ac{SMS} in general) to adopt, thus facilitating a rapid uptake into that community for application portability.
Both client and server \ac{PMIx} libraries are included, along with examples of client usage and server-side integration.}
\deleted{A list of supported environments and versions is maintained on the \ac{PMIx} web site \url{https://pmix.org/support/faq/what-apis-are-supported-on-my-rm/}}
\end{itemize}

\deleted{The \ac{PRI} does provide some internal implementations that lie outside the scope of the \ac{PMIx} standard. This includes several convenience macros as well as support for consolidating collectives for optimization purposes (e.g., the \ac{PMIx} server aggregates all local \refapi{PMIx_Fence} calls before
passing them to the \ac{SMS} for global execution). In a few additional cases, the \ac{PMIx} community (in partnership with the \ac{SMS} subsystem providers) have determined that a base level of support for a given operation can best be portably provided by including it in the \ac{PRI}.}

\deleted{Instructions for downloading, and installing the \ac{PRI} are available
on the community's web site \url{https://pmix.org/code/getting-the-reference-implementation/}.The \ac{PRI} targets support for the Linux operating system.
A reasonable effort is made to support all major, modern Linux distributions; however, validation is limited to the most recent 2-3 releases of RedHat Enterprise Linux (RHEL), Fedora, CentOS, and SUSE Linux Enterprise Server (SLES).
In addition, development support is maintained for Mac OSX.
Production support for vendor-specific operating systems is included as provided by the vendor.  }


%%%%%%%%%%%
\subsection{\deleted{The PMIx Reference RunTime Environment (PRRTE)}}

\deleted{The \ac{PMIx} community has also released \ac{PRRTE} --- i.e., a runtime environment
containing the reference implementation and capable of operating within a host \ac{SMS}. \ac{PRRTE}
provides an easy way of exploring \ac{PMIx} capabilities and testing PMIx-based
applications outside of a PMIx-enabled environment by providing a ``shim'' between the application and the host environment that includes full support for the \ac{PRI}. The intent of \ac{PRRTE} is not to replace any existing production environment, but rather to enable developers to work on systems that do not yet feature a PMIx-enabled host \ac{SMS} or one that lacks a \ac{PMIx} feature of interest. Instructions for downloading,
installing, and using \ac{PRRTE} are available
on the community's web site \url{https://pmix.org/code/getting-the-pmix-reference-server/}}

%%%%%%%%%%%%%%%%%%%%%%%%%%%%%%%%%%%%%%%%%%%%%%%%%
\section{Organization of this document}

The remainder of this document is structured as follows:

\begin{itemize}
\item Introduction and Overview in \chapterref{chap:intro}
\item Terms and Conventions in \chapterref{chap:terms}
\item Data Structures and Types in \chapterref{chap:struct}
\item \ac{PMIx} Initialization and Finalization in \chapterref{chap:api_init}
\item Key/Value Management in \chapterref{chap:api_kv_mgmt}
\item Process Management in \chapterref{chap:api_proc_mgmt}
\item Job Management in \chapterref{chap:api_job_mgmt}
\item Event Notification in \chapterref{chap:api_event}
\item Data Packing and Unpacking in \chapterref{chap:api_data_mgmt}
\item \ac{PMIx} Server Specific Interfaces in \chapterref{chap:api_server}
\end{itemize}

%%%%%%%%%%%%%%%%%%%%%%%%%%%%%%%%%%%%%%%%%%%%%%%%%
%%%%%%%%%% History: Version 1.0
\section{Version 1.0: June 12, 2015}

\par
The \ac{PMIx} version 1.0 \textit{ad hoc} standard was defined in the \acf{PRI} header files as part of the \ac{PRI} v1.0.0 release prior to the creation of the formal \ac{PMIx} 2.0 standard.
Below are a summary listing of the interfaces defined in the 1.0 headers.

\begin{itemize}
\item Client APIs
\begin{itemize}
\item PMIx\_Init, \refapi{PMIx_Initialized}, \refapi{PMIx_Abort}, \refapi{PMIx_Finalize}
\item \refapi{PMIx_Put}, \refapi{PMIx_Commit},
\item \refapi{PMIx_Fence}, \refapi{PMIx_Fence_nb}
\item \refapi{PMIx_Get}, \refapi{PMIx_Get_nb}
\item \refapi{PMIx_Publish}, \refapi{PMIx_Publish_nb}
\item \refapi{PMIx_Lookup}, \refapi{PMIx_Lookup}
\item \refapi{PMIx_Unpublish}, \refapi{PMIx_Unpublish_nb}
\item \refapi{PMIx_Spawn}, \refapi{PMIx_Spawn_nb}
\item \refapi{PMIx_Connect}, \refapi{PMIx_Connect_nb}
\item \refapi{PMIx_Disconnect}, \refapi{PMIx_Disconnect_nb}
\item \refapi{PMIx_Resolve_nodes}, \refapi{PMIx_Resolve_peers}
\end{itemize}
\item Server APIs
\begin{itemize}
\item \refapi{PMIx_server_init}, \refapi{PMIx_server_finalize}
\item \refapi{PMIx_generate_regex}, \refapi{PMIx_generate_ppn}
\item \refapi{PMIx_server_register_nspace}, \refapi{PMIx_server_deregister_nspace}
\item \refapi{PMIx_server_register_client}, \refapi{PMIx_server_deregister_client}
\item \refapi{PMIx_server_setup_fork}, \refapi{PMIx_server_dmodex_request}
\end{itemize}
\item Common APIs
\begin{itemize}
\item \refapi{PMIx_Get_version}, \refapi{PMIx_Store_internal}, \refapi{PMIx_Error_string}
\item PMIx_Register_errhandler, PMIx_Deregister_errhandler, PMIx_Notify_error
\end{itemize}
\end{itemize}

The \code{PMIx_Init} \ac{API} was subsequently modified in the \ac{PRI} release v1.1.0.

%%%%%%%%%%%%%%%%%%%%%%%%%%%%%%%%%%%%%%%%%%%%%%%%%
%%%%%%%%%% History: Version 2.0
\section{Version 2.0: Sept. 2018}

The following \acp{API} were introduced in v2.0 of the PMIx Standard:

\begin{itemize}
\item Client APIs
\begin{itemize}
\item \refapi{PMIx_Query_info_nb}, \refapi{PMIx_Log_nb}
\item \refapi{PMIx_Allocation_request_nb}, \refapi{PMIx_Job_control_nb}, \refapi{PMIx_Process_monitor_nb}, \refmacro{PMIx_Heartbeat}
\end{itemize}
\item Server APIs
\begin{itemize}
\item \refapi{PMIx_server_setup_application}, \refapi{PMIx_server_setup_local_support}
\end{itemize}
\item Tool APIs
\begin{itemize}
\item \refapi{PMIx_tool_init}, \refapi{PMIx_tool_finalize}
\end{itemize}
\item Common APIs
\begin{itemize}
\item \refapi{PMIx_Register_event_handler}, \refapi{PMIx_Deregister_event_handler}
\item \refapi{PMIx_Notify_event}
\item \refapi{PMIx_Proc_state_string}, \refapi{PMIx_Scope_string}
\item \refapi{PMIx_Persistence_string}, \refapi{PMIx_Data_range_string}
\item \refapi{PMIx_Info_directives_string}, \refapi{PMIx_Data_type_string}
\item \refapi{PMIx_Alloc_directive_string}
\item \refapi{PMIx_Data_pack}, \refapi{PMIx_Data_unpack}, \refapi{PMIx_Data_copy}
\item \refapi{PMIx_Data_print}, \refapi{PMIx_Data_copy_payload}
\end{itemize}
\end{itemize}

The \refapi{PMIx_Init} \ac{API} was modified in v2.0 of the standard from its \textit{ad hoc} v1.0 signature to include passing of a \refstruct{pmix_info_t} array for flexibility and ``future-proofing'' of the \ac{API}.
In addition, the PMIx_Notify_error, PMIx_Register_errhandler, and PMIx_Deregister_errhandler \acp{API} were replaced.

%%%%%%%%%%%%%%%%%%%%%%%%%%%%%%%%%%%%%%%%%%%%%%%%%
%%%%%%%%%% History: Version 2.1
\section{Version 2.1: Dec. 2018}

The v2.1 update includes clarifications and corrections from the v2.0 document, plus addition of examples:

\begin{itemize}
    \item Clarify description of \refapi{PMIx_Connect} and \refapi{PMIx_Disconnect} \acp{API}.
    \item Explain that values for the \refattr{PMIX_COLLECTIVE_ALGO} are environment-dependent
    \item Identify the namespace/rank values required for retrieving attribute-associated information using the \refapi{PMIx_Get} \ac{API}
    \item Provide definitions for \refterm{session}, \refterm{job}, \refterm{application}, and other terms used throughout the document
    \item Clarify definitions of \refattr{PMIX_UNIV_SIZE} versus \refattr{PMIX_JOB_SIZE}
    \item Clarify server module function return values
    \item Provide examples of the use of \refapi{PMIx_Get} for retrieval of information
    \item Clarify the use of \refapi{PMIx_Get} versus \refapi{PMIx_Query_info_nb}
    \item Clarify return values for non-blocking \acp{API} and emphasize that callback functions must not be invoked prior to return from the \ac{API}
    \item Provide detailed example for construction of the \refapi{PMIx_server_register_nspace} input information array
    \item Define information levels (e.g., \refterm{session} vs \refterm{job}) and associated attributes for both storing and retrieving values
    \item Clarify roles of \ac{PMIx} server library and host environment for collective operations
    \item Clarify definition of \refattr{PMIX_UNIV_SIZE}
\end{itemize}

%%%%%%%%%%%%%%%%%%%%%%%%%%%%%%%%%%%%%%%%%%%%%%%%%
%%%%%%%%%% History: Version 2.2
\section{Version 2.2: Jan 2019}

The v2.2 update includes the following clarifications and corrections from the v2.1 document:

\begin{itemize}
    \item Direct modex upcall function (\refapi{pmix_server_dmodex_req_fn_t}) cannot complete atomically as the \ac{API} cannot return the requested information except via the provided callback function
    \item Add missing \refstruct{pmix_data_array_t} definition and support macros
    \item Add a rule divider between implementer and host environment required attributes for clarity
    \item Add \refmacro{PMIX_QUERY_QUALIFIERS_CREATE} macro to simplify creation of \refstruct{pmix_query_t} qualifiers
    \item Add \refmacro{PMIX_APP_INFO_CREATE} macro to simplify creation of \refstruct{pmix_app_t} directives
    \item Add flag and \refmacro{PMIX_INFO_IS_END} macro for marking and detecting the end of a \refstruct{pmix_info_t} array
    \item Clarify the allowed hierarchical nesting of the \refattr{PMIX_SESSION_INFO_ARRAY}, \refattr{PMIX_JOB_INFO_ARRAY}, and associated attributes
\end{itemize}

%%%%%%%%%%%%%%%%%%%%%%%%%%%%%%%%%%%%%%%%%%%%%%%%%
%%%%%%%%%% History: Version 3.0
\section{Version 3.0: Dec. 2018}

The following \acp{API} were introduced in v3.0 of the PMIx Standard:

\begin{itemize}
\item Client APIs
\begin{itemize}
\item \refapi{PMIx_Log}, \refapi{PMIx_Job_control}
\item \refapi{PMIx_Allocation_request}, \refapi{PMIx_Process_monitor}
\item \refapi{PMIx_Get_credential}, \refapi{PMIx_Validate_credential}
\end{itemize}
\item Server APIs
\begin{itemize}
\item \refapi{PMIx_server_IOF_deliver}
\item \refapi{PMIx_server_collect_inventory}, \refapi{PMIx_server_deliver_inventory}
\end{itemize}
\item Tool APIs
\begin{itemize}
\item \refapi{PMIx_IOF_pull}, \refapi{PMIx_IOF_push}, \refapi{PMIx_IOF_deregister}
\item \refapi{PMIx_tool_connect_to_server}
\end{itemize}
\item Common APIs
\begin{itemize}
\item \refapi{PMIx_IOF_channel_string}
\end{itemize}
\end{itemize}

The document added a chapter on security credentials, a new section for \ac{IO} forwarding to the Process Management chapter, and a few blocking forms of previously-existing non-blocking \acp{API}. Attributes supporting the new \acp{API} were introduced, as well as additional attributes for a few existing functions.

%%%%%%%%%%%%%%%%%%%%%%%%%%%%%%%%%%%%%%%%%%%%%%%%%
%%%%%%%%%% History: Version 3.1
\section{Version 3.1: Jan. 2019}

The v3.1 update includes clarifications and corrections from the v3.0 document:

\begin{itemize}
    \item Direct modex upcall function (\refapi{pmix_server_dmodex_req_fn_t}) cannot complete atomically as the \ac{API} cannot return the requested information except via the provided callback function
    \item Fix typo in name of \refattr{PMIX_FWD_STDDIAG} attribute
    \item Correctly identify the information retrieval and storage attributes as ``new'' to v3 of the standard
    \item Add missing \refstruct{pmix_data_array_t} definition and support macros
    \item Add a rule divider between implementer and host environment required attributes for clarity
    \item Add \refmacro{PMIX_QUERY_QUALIFIERS_CREATE} macro to simplify creation of \refstruct{pmix_query_t} qualifiers
    \item Add \refmacro{PMIX_APP_INFO_CREATE} macro to simplify creation of \refstruct{pmix_app_t} directives
    \item Add new attributes to specify the level of information being requested where ambiguity may exist (see \ref{api:struct:attributes:retrieval})
    \item Add new attributes to assemble information by its level for storage where ambiguity may exist (see \ref{api:struct:attributes:storage})
    \item Add flag and \refmacro{PMIX_INFO_IS_END} macro for marking and detecting the end of a \refstruct{pmix_info_t} array
    \item Clarify that \code{PMIX_NUM_SLOTS} is duplicative of (a) \refattr{PMIX_UNIV_SIZE} when used at the \refterm{session} level and (b) \refattr{PMIX_MAX_PROCS} when used at the \refterm{job} and \refterm{application} levels, but leave it in for backward compatibility.
    \item Clarify difference between \refattr{PMIX_JOB_SIZE} and \refattr{PMIX_MAX_PROCS}
    \item Clarify that \refapi{PMIx_server_setup_application} must be called per-\refterm{job} instead of per-\refterm{application} as the name implies. Unfortunately, this is a historical artifact. Note that both \refattr{PMIX_NODE_MAP} and \refattr{PMIX_PROC_MAP} must be included as input in the \refarg{info} array provided to that function. Further descriptive explanation of the ``instant on'' procedure will be provided in the next version of the \ac{PMIx} Standard.
    \item Clarify how the \ac{PMIx} server expects data passed to the host by \refapi{pmix_server_fencenb_fn_t} should be aggregated across nodes, and provide a code snippet example
\end{itemize}


%%%%%%%%%%%%%%%%%%%%%%%%%%%%%%%%%%%%%%%%%%%%%%%%%
%%%%%%%%%% History: Version 4.0
\section{Version 4.0: June 2019}

The following changes were introduced in v4.0 of the PMIx Standard:

\begin{itemize}
    \item Clarified that the \refapi{PMIx_Fence_nb} operation can immediately return \refconst{PMIX_OPERATION_SUCCEEDED} in lieu of passing the request to a \ac{PMIx} server if only the calling process is involved in the operation
    \item Added the \refapi{PMIx_Register_attributes} \ac{API} by which a host environment can register the attributes it supports for each server-to-host operation
    \item Added the ability to query supported attributes from the \ac{PMIx} tool, client and server libraries, as well as the host environment via the new \refstruct{pmix_regattr_t} structure. Both human-readable and machine-parsable output is supported. New attributes to support this operation include:
    \begin{itemize}
        \item \refattr{PMIX_CLIENT_ATTRIBUTES}, \refattr{PMIX_SERVER_ATTRIBUTES}, \refattr{PMIX_TOOL_ATTRIBUTES}, and \refattr{PMIX_HOST_ATTRIBUTES} to identify which library supports the attribute; and
        \item \refattr{PMIX_MAX_VALUE}, \refattr{PMIX_MIN_VALUE}, and \refattr{PMIX_ENUM_VALUE} to provide machine-parsable description of accepted values
    \end{itemize}
\end{itemize}
